\begin{lstlisting}[title=\href{https://godbolt.org/z/nv83nh}{\texttt{godbolt.org/z/nv83nh}}]
#include <iostream>

int main() {
    int i = 1;
    auto x = [i]() { return ++i; };
    std::cout << i << x() << i;
}
\end{lstlisting}
