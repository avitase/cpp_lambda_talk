\section{Recursion}
Implement a lambda expression \texttt{f(int x)} that calculates $x!$ (factorial of \texttt{x}) by recursion upon calling!
Use the following skeleton:
\begin{lstlisting}
int main() {
    auto f = /* TODO */
        return x < 2 ? x : x * f( /* TODO */ );
    });

    return f(3);
}
\end{lstlisting}

\section{Function traits}
\label{sec:fcttraits}
Finish the implementation of \texttt{function\_traits} and \texttt{lambda\_traits} such that the program compiles successfully! Note that it is sufficient for \texttt{function\_traits} to only accept instances of \texttt{std::function}. Think about how you can use your implementation of \texttt{function\_traits} with lambda expressions and implement this logic in \texttt{lambda\_traits}. Note further that invocations of \texttt{std::function} are not critical in terms of possible performance penalties in this scenario, as long as the traits are available at compile time (as required in \texttt{main}).

\inputcpplisting{snippet21a}

\textbf{Hint:} You may want to use \texttt{std::tuple\_element} and \texttt{std::declval}.

\newpage

\section{Passing lambdas as argument}
Implement \texttt{Filters} as an logical \textit{or} filter gathering such that the program below compiles and runs successfully.

\inputcpplisting{snippet20a}

The constructor of \texttt{Filters} should take one or more filters,
\begin{lstlisting}
template <typename F, typename... Fs> Filters(F f, Fs... fs)
\end{lstlisting}
where each filter in \texttt{fs} (and \texttt{f} itself) is a lambda (or \texttt{std::function}) that accepts an argument of type \texttt{T} and returns a boolean. \texttt{T} should be generic and is allow to be different amongst the filters.
The call operator 
\begin{lstlisting}
bool Filters::operator()(T x) const
\end{lstlisting}
should return the logical \textit{or} of all filters for applied \texttt{x} as shown in the example.
Further, implement a function
\begin{lstlisting}
template <typename F> void Filters::add_filter(F)
\end{lstlisting}
that adds a filter to the gathering. Note that in the example there is no common type in terms of \texttt{std::common\_type\_t} for \texttt{f1}, \texttt{f2} and \texttt{f3}:
\begin{lstlisting}
using T12 = std::common_type_t<decltype(f1), decltype(f2)>; // OK
using T123 = std::common_type_t<T12, decltype(f3)>;         // Compile-time error!
\end{lstlisting}

If in trouble, follow the step-by-step instructions printed below!

\subsection{Step I}
Start by finding an implementation for \texttt{Filters} that suffices the relaxed requirement
\begin{lstlisting}
template <typename F1, typename F2>
struct Filters {
    Filters(F1 f1, F2 f2) {
        /* TODO */
    }

    [[nodiscard]] auto operator()(int x) const {
        /* TODO */
    }
};

int main() {
    auto f1 = [](int x) { return x % 2 == 0; };
    auto f2 = [](int x) { return x % 3 == 0; };
    Filters filters(f1, f2);
    return filters(5) ? 1 : 0;
}
\end{lstlisting}
\ldots and answer the question why we need two separate template types \texttt{F1} and \texttt{F2}!

\subsection{Step II}
Make the type of the passed argument of a filter (\texttt{int}) generic and pass it as a template parameter. You will now need to instantiate \texttt{Filters} akin to
\begin{lstlisting}
auto f1 = [](int x) { return x % 2 == 0; };
auto f2 = [](int x) { return x % 3 == 0; };
Filters<decltype(f1), decltype(f2), int> filters(f1, f2);
\end{lstlisting}
This is unpleasent and we will address this issue in the next step.

\subsection{Step III}
Use a deduction guide to get rid of the explicit type naming of the filters. This is a delicate task, since we have to find the type of the argument of a passed filter. You may want to have a look at Sec.~\ref{sec:fcttraits} to find a solution for this problem.
\begin{lstlisting}
template <typename F1, typename F2>
Filters(F1 f1, F2 f2) -> Filters<F1, F2, std::common_type_t< /* TODO */ >>
\end{lstlisting}

\subsection{Step IV}
We now generalize our solution for an arbitrary number of filters. Use variadic templates for the constructor and the deduction guide, and store the filters in a \texttt{std::vector}! Use \texttt{std::function<bool(T)>} as the value type of the vector.

\begin{lstlisting}
template <typename T>
struct Filters {
    std::vector<std::function<bool(T)>> filters;

    template <typename F, typename... Fs>
    Filters(F f, Fs... fs): filters( /* TODO */ ) {}
    
    /* TODO */
};

template <typename... Fs>
Filters(Fs... fs)
-> Filters<std::common_type_t< /* TODO */ >>;
\end{lstlisting}
Add a sufficient implementation of
\begin{lstlisting}
template <typename F> void Filters::add_filter(F)
\end{lstlisting}
and don't forget to adopt your implementation of \texttt{Filters.operator()(T)}! (use \texttt{std::accumulate} if possible.)

\subsection{Step V}
Can you think of an alternative to \texttt{std::function}? Why can we not use \texttt{std::common\_type} or \texttt{std::variant}?

\textbf{Hint:} Compare the sizes of lambdas with different captures:
\begin{lstlisting}[title=\href{https://godbolt.org/z/ZBUfgY}{\texttt{godbolt.org/z/ZBUfgY}}]
int capture_me = 1;
std::cout << sizeof([](int x) { return x; }) << '\n';
std::cout << sizeof([y=capture_me](int x) { return x + y; }) << '\n';
std::cout << sizeof([y=&capture_me](int x) { return x + y; }) << '\n';
\end{lstlisting}

\subsection{Optional}
In case you are wondering what \texttt{std::common\_type} does; its rules are based on the rules for the ternary operator which can be confusing, e.g.
\begin{lstlisting}[title=\href{https://godbolt.org/z/4TGmK6}{\texttt{godbolt.org/z/4TGmK6}}]
struct S {};

template<class T, int> struct CT {
    operator T() const;
};

int main() {
    auto a = false ? CT<int, 1>{} : CT<int, 2>{};   // OK
    auto b = false ? CT<int*, 1>{} : CT<int*, 2>{}; // OK
    auto c = false ? CT<S*, 1>{} : CT<S*, 2>{};     // OK
    auto d = false ? CT<S, 1>{} : CT<S, 2>{};       // Compile-time error
}
\end{lstlisting}
Depending on what compiler you are using, the error message for \texttt{d} varies. Take a look at:
\begin{itemize}
    \item \texttt{[over.build] \S27}
    \item \texttt{[expr.cond] \S6}
\end{itemize}
if you want to learn more.
