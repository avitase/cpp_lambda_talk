\begin{frame}[plain,noframenumbering]
    \centering
    \scalebox{3}{Introduction}
\end{frame}

\section{Introduction}

\begin{frame}[fragile]{What is a Lambda Expression in \texttt{C++}?}
    \texttt{cube} is a lambda \ldots
    \inputcpplisting{snippet1}

    \texttt{is\_even} is a lambda \ldots
    \inputcpplisting{snippet2}
\end{frame}

\begin{frame}[plain,noframenumbering]
    \centering
    \scalebox{3}{Syntax}
\end{frame}

\section{Syntax}

\begin{frame}[fragile]{\texttt{C++}'s Lambda Expression}
    The simplest (and most boring) lambda
    \begin{lstlisting}
auto x = []{};
    \end{lstlisting}
    \hfill \ldots no capturing, takes no parameters and returns nothing

    \vspace{5mm}

    \begin{columns}[t]
        \begin{column}{.45\textwidth}
            A slightly more \enquote{useful} lambda
            \inputcpplisting{snippet3}
        \end{column}
        \begin{column}{.45\textwidth}
            \ldots is equivalent to
            \inputcpplisting{snippet4}
        \end{column}
    \end{columns}
\end{frame}

\begin{frame}[fragile]{Q: What is the output of the program?}
    \begin{lstlisting}[language=python]
#!/usr/bin/env python3

if __name__ == '__main__':
    f = {k: lambda x: x + k for k in range(3)}

    for k in range(3):
        print(f[k](2), end='')
    \end{lstlisting}
\end{frame}

% REMOVE_FOR_PRINT {
\addtocounter{framenumber}{-1}
\begin{frame}[fragile]{A: \texttt{444}}
    \begin{lstlisting}[language=python]
#!/usr/bin/env python3

if __name__ == '__main__':
    f = {k: lambda x: x + k for k in range(3)}

    for k in range(3):
        print(f[k](2), end='')
    \end{lstlisting}

    \hfill \textbf{Why?} implicit capture by reference
\end{frame}
% REMOVE_FOR_PRINT }

\begin{frame}[fragile]{Fix}
    \begin{lstlisting}[language=python]
#!/usr/bin/env python3

from functools import partial

if __name__ == '__main__':
    f = {k: partial(lambda x, k: x + k, k=k) for k in range(3)}
    # f = {k: lambda x, k=k: x + k for k in range(3)}
    # ... would change API

    for k in range(3):
        print(f[k](2), end='')
    \end{lstlisting}
    
    Now prints: \texttt{234}
\end{frame}

\begin{frame}[fragile]{Q: What is the output of the program?}
    \inputcpplisting{snippet34a}

% REMOVE_FOR_PRINT {
    \only<2>{\textbf{\textcolor{vertexDarkRed}{error:}} \enquote{k} is not captured}
% REMOVE_FOR_PRINT }
\end{frame}

\begin{frame}[fragile]{Q: What is the output of the program?}
    \inputcpplisting{snippet34b}
\end{frame}

% REMOVE_FOR_PRINT {
\addtocounter{framenumber}{-1}
\begin{frame}[fragile]{A: \text{234}}
    \inputcpplisting{snippet34b}
\end{frame}
% REMOVE_FOR_PRINT }

\begin{frame}[fragile]{Q: What is the output of the program?}
    \inputcpplisting{snippet34c}
\end{frame}

% REMOVE_FOR_PRINT {
\addtocounter{framenumber}{-1}
\begin{frame}[fragile]{A: \texttt{555} (not \texttt{444})}
    \inputcpplisting{snippet34c}
\end{frame}
% REMOVE_FOR_PRINT }

\begin{frame}[fragile]{\texttt{C++}'s Lambda Expression}
    \textbf{Capturing rules}
    \begin{itemize}
        \item \texttt{[x]}: captures \texttt{x} by value
        \item \texttt{[\&x]}: captures \texttt{x} by reference
        \item \texttt{[=]}: captures all variables (used in the lambda) by value
        \item \texttt{[\&]}: captures all variables (used in the lambda) by reference
        \item \texttt{[=, \&x]}: captures variables like with \texttt{[=]}, but \texttt{x} by reference
        \item \texttt{[\&, x]}: captures variables like with \texttt{[\&]}, but \texttt{x} by value
    \end{itemize}
\end{frame}

\begin{frame}[fragile]{Capturing by value}
    \begin{columns}[t]
        \begin{column}{.45\textwidth}
            \inputcpplisting{snippet5}
        \end{column}
        \begin{column}{.45\textwidth}
            \ldots or equivalently
            \inputcpplisting{snippet6}
        \end{column}
    \end{columns}
\end{frame}

\begin{frame}[fragile]{Capturing by reference}
    \begin{columns}[t]
        \begin{column}{.45\textwidth}
            \inputcpplisting{snippet7}
        \end{column}
        \begin{column}{.45\textwidth}
            \ldots or equivalently
            \inputcpplisting{snippet8}
        \end{column}
    \end{columns}
\end{frame}

\begin{frame}[fragile]{Q: What is the output of the program?}
    \inputcpplisting{snippet9a}

% REMOVE_FOR_PRINT {
    \only<2>{\textbf{\textcolor{vertexDarkRed}{error:}} cannot assign to a variable captured by copy in a non-mutable lambda}
% REMOVE_FOR_PRINT }
\end{frame}

\begin{frame}[fragile]{Q: What is the output of the program?}
    \inputcpplisting{snippet9b}
\end{frame}

% REMOVE_FOR_PRINT {
\addtocounter{framenumber}{-1}
\begin{frame}[fragile]{A: \texttt{121}}
    \inputcpplisting{snippet9b}
\end{frame}
% REMOVE_FOR_PRINT }

\begin{frame}[fragile]{Q: What is the output of the program?}
    \inputcpplisting{snippet9c}
\end{frame}

% REMOVE_FOR_PRINT {
\addtocounter{framenumber}{-1}
\begin{frame}[fragile]{A: \texttt{122}}
    \inputcpplisting{snippet9c}
\end{frame}
% REMOVE_FOR_PRINT }

\begin{frame}[fragile]{Q: What is the output of the program?}
    \inputcpplisting{snippet9d}
\end{frame}

% REMOVE_FOR_PRINT {
\addtocounter{framenumber}{-1}
\begin{frame}[fragile]{A: \texttt{12}}
    \inputcpplisting{snippet9d}
\end{frame}
% REMOVE_FOR_PRINT }

\begin{frame}[fragile]{Q: What is the output of the program?}
    \inputcpplisting{snippet10}

    \footnotesize
    \begin{center}
    \href{https://en.cppreference.com/w/cpp/utility/exchange}{\texttt{cppreference.com/w/cpp/utility/exchange}} \\
    or \\
    \href{https://www.fluentcpp.com/2020/09/25/stdexchange-patterns-fast-safe-expressive-and-probably-underused/}{Ben Deane, \textit{std::exchange Patterns: Fast, Safe, Expressive, and Probably Underused} on \texttt{fluentcpp.com}}
    \end{center}
\end{frame}

% REMOVE_FOR_PRINT {
\addtocounter{framenumber}{-1}
\begin{frame}[fragile]{A: \texttt{11235}}
    \inputcpplisting{snippet10}

    \footnotesize
    \begin{center}
    \href{https://en.cppreference.com/w/cpp/utility/exchange}{\texttt{cppreference.com/w/cpp/utility/exchange}} \\
    or \\
    \href{https://www.fluentcpp.com/2020/09/25/stdexchange-patterns-fast-safe-expressive-and-probably-underused/}{Ben Deane, \textit{std::exchange Patterns: Fast, Safe, Expressive, and Probably Underused} on \texttt{fluentcpp.com}}
    \end{center}
\end{frame}
% REMOVE_FOR_PRINT }

\begin{frame}{\texttt{C++}'s Lambda Expression}
    Remember, lambda expressions are pure syntactic sugar and are equivalent to \texttt{struct}s with an appropriate \texttt{operator()()} overload \ldots
\end{frame}

\begin{frame}[fragile]{Q: What is the output of the program?}
    \inputcpplisting{snippet11}
\end{frame}

% REMOVE_FOR_PRINT {
\addtocounter{framenumber}{-1}
\begin{frame}[fragile]{A: \texttt{11}}
    \inputcpplisting{snippet11}
\end{frame}
% REMOVE_FOR_PRINT }

\begin{frame}[fragile]{Q: What is the output of the program?}
    \inputcpplisting{snippet12}
\end{frame}

% REMOVE_FOR_PRINT {
\addtocounter{framenumber}{-1}
\begin{frame}[fragile]{A: \texttt{66}}
    \inputcpplisting{snippet12}
\end{frame}
% REMOVE_FOR_PRINT }

\begin{frame}[fragile]{Q: What is the output of the program?}
    \inputcpplisting{snippet13}

% REMOVE_FOR_PRINT {
    \only<2>{\textbf{\textcolor{vertexDarkRed}{error:}} call to implicitly-deleted copy ctor}
% REMOVE_FOR_PRINT }
\end{frame}

\begin{frame}[plain,noframenumbering]
    \centering
    \scalebox{3}{Stateful Lambdas}
\end{frame}

\section{Stateful Lambdas}

\begin{frame}[fragile]{Q: What is the output of the program?}
    \inputcpplisting{snippet14}
\end{frame}

% REMOVE_FOR_PRINT {
\addtocounter{framenumber}{-1}
\begin{frame}[fragile]{A: \texttt{3}}
    \inputcpplisting{snippet14}
\end{frame}
% REMOVE_FOR_PRINT }

\begin{frame}[fragile]{Q: What is the output of the program?}
    \inputcpplisting{snippet15}
\end{frame}

% REMOVE_FOR_PRINT {
\addtocounter{framenumber}{-1}
\begin{frame}[fragile]{A: \texttt{5}}
    \inputcpplisting{snippet15}

    \begin{center}
        ($^*$ undefined in a threaded context, since \texttt{static} is not thread-safe!)
    \end{center}
\end{frame}
% REMOVE_FOR_PRINT }

\begin{frame}[fragile]{Stateful Lambdas}
    \textbf{Fibonacci (again)}:
    \inputcpplisting{snippet16a}
\end{frame}

\begin{frame}[fragile]{Stateful Lambdas}
    Let us now try to interact with the state of the Lambda \ldots
    \inputcpplisting{snippet16b}
\end{frame}

\begin{frame}[fragile]{Stateful Lambdas}
    \ldots or slightly more conveniently:
    \inputcpplisting{snippet16c}
\end{frame}

\begin{frame}[fragile]{Stateful Lambdas}
    \begin{columns}[t]
        \begin{column}{.6\textwidth}
            \inputcpplisting{snippet16d}
        \end{column}
        \begin{column}{.35\textwidth}
            \inputasmlisting{snippet16d}
        \end{column}
    \end{columns}
\end{frame}

\begin{frame}[plain,noframenumbering]
    \centering
    \scalebox{3}{Best Practices}

    \scalebox{.8}{(partially taken from \enquote{Effective Modern C++} by Scott Meyers)}
\end{frame}

\section{Best Practices}

\begin{frame}[fragile]{Use Lambdas in STL algorithm}
    \inputcpplisting{snippet17a}
\end{frame}

\begin{frame}[fragile]{Use Lambdas in STL algorithm}
    \inputcpplisting{snippet17b}
\end{frame}

\begin{frame}[fragile]{Stop pollution of namespace with helper variables}
    \inputcpplisting{snippet18}
\end{frame}

\begin{frame}[fragile]{Allow variables to be \texttt{const}}
    \inputcpplisting{snippet19}

    \begin{center}
        \footnotesize
        (cf.\ \textbf{\textit{IIFE}}: \href{https://www.bfilipek.com/2016/11/iife-for-complex-initialization.html}{\texttt{bfilipek.com/2016/11/iife-for-complex-initialization.html}} \\ or \\ \href{https://foonathan.net/2020/10/iife-metaprogramming/}{\texttt{foonathan.net/2020/10/iife-metaprogramming/}})
    \end{center}
\end{frame}

\begin{frame}[fragile]{Avoid default capture modes}
    Below, there is a dangling pointer lurking in the wings \ldots
    \begin{lstlisting}
void add_filter() {
    auto divisor = get_magic_number();
    filters.emplace_back([&](int x) { return x % divisor == 0; });
}
    \end{lstlisting}

    This error becomes more obvious, when explicit capturing is used:
    \begin{lstlisting}
void add_filter() {
    auto divisor = get_magic_number();
    filters.emplace_back([&divisor](int x) { return x % divisor == 0; });
}
    \end{lstlisting}
\end{frame}

\begin{frame}[fragile]{Avoid default capture modes}
    Mitigation of copy \& paste bugs:
    \begin{lstlisting}
auto divisor = get_magic_number();
std::find_if(container.begin(),
             container.end(),
             [&divisor](int x) { return x % divisor == 0; });
    \end{lstlisting}

    \texttt{[\&divisor]} indicates that there is an \textit{external} dependency and it is not enough to \enquote{just copy} the lambda function if needed elsewhere.

    {\footnotesize (off-topic: check out this interesting article about copy \& paste bugs in real world applications: \enquote{The Last Line Effect} by the PVS-Studio team, \href{https://www.viva64.com/en/b/0260/}{\texttt{www.viva64.com/en/b/0260/}})}  
\end{frame}

\begin{frame}[fragile]{Avoid default capture modes}
    Does the following implementation looks fine?
    \begin{lstlisting}
struct Widget {
    int divisor = 2;

    void add_filter() const {
        filters.emplace_back([=](int x) { return x % divisor == 0; });
    }
};
    \end{lstlisting}
    \hfill \ldots given a sufficient implementation of \texttt{filters}

% REMOVE_FOR_PRINT {
    \only<2>{%
% REMOVE_FOR_PRINT }
    \scalebox{1.2}{\textbf{No! Horrible code!}}
    Capturing only applies to non-\texttt{static} local variables. Why does this work?
% REMOVE_FOR_PRINT {
    }
% REMOVE_FOR_PRINT }
\end{frame}

\begin{frame}[fragile]{Avoid default capture modes}
    Capturing only applies to non-\texttt{static} local variables. Why does this work?
    \begin{lstlisting}
Widget::add_filter() const {
    filters.emplace_back([=](int x) { return x % divisor == 0; });
}
    \end{lstlisting}
    \ldots but this fails
    \begin{lstlisting}
Widget::add_filter() const {
    filters.emplace_back([](int x) { return x % divisor == 0; });
}
    \end{lstlisting}

    \ldots and this also
    \begin{lstlisting}
Widget::add_filter() const {
    filters.emplace_back([divisor](int x) { return x % divisor == 0; });
}
    \end{lstlisting}
\end{frame}

\begin{frame}[fragile]{Avoid default capture modes}
    There is no local variable \texttt{divisor}! But what happes is the following
    \begin{lstlisting}
Widget::add_filter() const {
    filters.emplace_back([=](int x) {
        return x % divisor == 0;
    });
}
    \end{lstlisting}
    copies (implicitly) \texttt{this} pointer (\textcolor{vertexDarkRed}{until \texttt{C++17}}), i.e.
    \begin{lstlisting}    
Widget::add_filter() const {
    auto copy_of_this = this;
    filters.emplace_back([copy_of_this](int x) {
        return x % copy_of_this->divisor == 0;
    });
}
    \end{lstlisting}

    \ldots welcome to the world of undefined behavior, when \texttt{Widget} goes out of scope!
\end{frame}

\begin{frame}[fragile]{Avoid default capture modes}
    Default capturing by value can be misleading and gives the impression that a lambda is self-contained:
    \begin{lstlisting}
static auto divisor = 1;
filters.emplace_back([=](int x) { return x % divisor == 0; });
++divisor;
    \end{lstlisting}
    Above, \texttt{divisor} is not copied! (as one may have guessed seeing \texttt{[=]})
\end{frame}

\begin{frame}[fragile]{Stop using \texttt{std::bind}}
    \scalebox{1.2}{Stop using \texttt{std::bind}}

    \ldots and prefer lambda expression, since
    \begin{itemize}
        \item this increases readability,
        \item lambdas are much more flexible,
        \item \texttt{std::bind} can potentially introduce additional overhead at run-time, whereas lambdas are default \texttt{constexpr}
    \end{itemize}
\end{frame}

\begin{frame}[fragile]{Stop using \texttt{std::function}}
    \scalebox{1.2}{Stop using \texttt{std::function}}

    \begin{itemize}
        \item \texttt{std::function} add multiple copies of passed object (consider using drop-in replacements such as \textit{delegate}s${}^{\color{vertexDarkRed}\star}$)
        \item may cause heap allocation
        \item is just a wrapper \ldots
    \end{itemize}
    \ldots deduce type of lambda via \texttt{auto} or template deduction, \textbf{if possible} (cf. exercise)

    \vspace{5mm}

    \scalebox{.75}{${}^{\color{vertexDarkRed}\star}$\href{https://codereview.stackexchange.com/questions/14730/impossibly-fast-delegate-in-c11}{\texttt{codereview.stackexchange.com/questions/14730/impossibly-fast-delegate-in-c11}}}
    
\end{frame}
